\documentclass[aspectratio=169]{beamer}

\usetheme{Madrid}
\usecolortheme{dolphin}

\usepackage[T1]{fontenc}
\usepackage[utf8]{inputenc}
\usepackage{lmodern}
\usepackage{graphicx}
\usepackage{booktabs}
\usepackage{amsmath,amssymb}
\usepackage{hyperref}

\title[RunTime]{RunTime: Distributional Transformers for Irregular Event Sequences}
\subtitle{A causal Transformer framework for distributional regression on event trajectories}
\author{Yael S. Elmatad}
\date{January 2026}

\newcommand{\repo}{\texttt{github.com/yaelelmatad/RunTime-Public}}

\begin{document}

\begin{frame}
  \titlepage
\end{frame}

\begin{frame}{The problem}
Many real-world systems are \textbf{event-driven}:
\begin{itemize}
  \item events arrive at \textbf{irregular intervals}
  \item each event has context (environment/state) and an outcome
  \item uncertainty matters (tails drive decisions)
\end{itemize}
\vspace{0.6em}
\textbf{Typical regression} flattens history into a table and predicts a point estimate, losing cadence and uncertainty.
\end{frame}

\begin{frame}{Key idea (one sentence)}
\textbf{Represent each event as a fixed token block}, quantize continuous variables into bins, treat \textbf{time deltas as tokens}, and train a \textbf{causal Transformer} to predict a \textbf{full PDF} over outcomes (not just a point).
\end{frame}

\begin{frame}{RunTime at a glance}
\begin{itemize}
  \item \textbf{Input}: tokenized trajectory (stride-based blocks) + irregular time tokens
  \item \textbf{Model}: masked self-attention (causal)
  \item \textbf{Output}: softmax distribution over discretized outcome bins
  \item \textbf{Training}: cross-entropy with \textbf{seconds-aware Gaussian smoothing} targets
\end{itemize}
\end{frame}

\begin{frame}{Why predict a PDF (not just a point estimate)?}
Point estimates hide risk:
\begin{itemize}
  \item multimodality and heavy tails collapse into a single number
  \item confidence becomes invisible
\end{itemize}
\vspace{0.6em}
RunTime outputs a \textbf{probability density function (PDF)} over discretized bins, enabling:
\begin{itemize}
  \item uncertainty-aware decisions
  \item median/mean/mode point predictions as summaries
  \item auditability via attention/activation inspection
\end{itemize}
\end{frame}

\begin{frame}{The 11-token event block (grammar)}
Each event is encoded as a fixed stride (11 tokens), mixing context, cadence, and target:
\begin{itemize}
  \item demographics: \texttt{age\_*}, \texttt{gen\_*}
  \item context: \texttt{cond\_*}, \texttt{hum\_*}, \texttt{temp\_*}, \texttt{feels\_*}, \texttt{wind\_*}
  \item distance: \texttt{distance\_name\_token\_*}
  \item outcome: \texttt{pace\_*} (the supervised target token)
  \item cadence: \texttt{d\_next\_*} (time since previous), \texttt{d\_fin\_*} (time to target)
\end{itemize}
\end{frame}

\begin{frame}{Example ``sentence'' (two-event trajectory)}
\small
Event 1 (history):
\vspace{-0.2em}
\begin{center}
\texttt{[age\_35, gen\_M, cond\_Clear, hum\_45, temp\_55, feels\_55, wind\_5,}\\
\texttt{distance\_name\_token\_10\_kilometers, pace\_115, d\_next\_12, d\_fin\_12]}
\end{center}
\vspace{0.3em}
Event 2 (target context, pace is predicted):
\vspace{-0.2em}
\begin{center}
\texttt{[age\_35, gen\_M, cond\_Rain, hum\_85, temp\_48, feels\_42, wind\_15,}\\
\texttt{distance\_name\_token\_half\_marathon, pace\_\textit{?}, d\_next\_0, d\_fin\_0]}
\end{center}
\normalsize
\vspace{0.3em}
The model conditions on the full token sequence and predicts a \textbf{distribution} over the final \texttt{pace} token.
\end{frame}

\begin{frame}{Week-delta tokens: injecting cadence into the sequence}
Instead of a single numeric feature, time is represented as tokens:
\begin{itemize}
  \item \texttt{d\_next}: time between adjacent events (cadence / recency)
  \item \texttt{d\_fin}: time remaining to the final target event (horizon)
\end{itemize}
\vspace{0.6em}
This makes irregularity explicit and lets attention learn non-linear decay / momentum effects.
\end{frame}

\begin{frame}{Missing data, imputation, and survival-style time-to-event}
\begin{itemize}
  \item Many forecasters assume a \textbf{regular grid}; irregularity is handled via \textbf{imputation/resampling}, which can \textbf{smear signal} and hide meaningful gaps.
  \item RunTime treats \textbf{irregular event time} as first-class input (time tokens), so ``no observation'' can remain a \textbf{gap} rather than a fabricated value.
  \item \textbf{Missingness tokens generalize beyond time}: you can represent absent covariates explicitly (e.g., \texttt{temp\_missing}, \texttt{humidity\_missing}) instead of imputing continuous values.
    \begin{itemize}
      \item This plays nicely with mixed representations: keep continuous-valued features continuous when present, but include a dedicated \texttt{*\_missing} token/state when absent.
    \end{itemize}
  \item For \textbf{time-to-next-event} tasks (survival/churn/retention), the discrete grammar can represent informative missingness:
    \begin{itemize}
      \item \texttt{time\_censored}: no future event within the observation window
      \item \texttt{time\_missing}: next-event timing unknown but the entity remains active
    \end{itemize}
  \item This enables \textbf{distributional} time-to-event prediction (e.g., ``time-to-next-admission'' PDFs) with explicit handling of censoring.
\end{itemize}
\end{frame}

\begin{frame}{Balanced quantization (binning strategy)}
Continuous variables are discretized into bins with a simple goal: \textbf{maximize usable signal per token}.
\begin{itemize}
  \item use \textbf{balanced} (approximately equal-mass) bins rather than equal-width bins
  \item yields higher entropy targets and avoids extremely sparse classes
  \item bin widths adapt: narrow where data is dense, wider where data is sparse
\end{itemize}
\vspace{0.6em}
The model predicts a PDF over pace bins; bin medians support mean/median/mode summaries.
\end{frame}

\begin{frame}{Quantization as regularization (and minimal tuning)}
Quantization also acts as a structural regularizer:
\begin{itemize}
  \item reduces sensitivity to micro-noise in seconds-level outcomes
  \item encourages learning transitions between meaningful states
\end{itemize}
\vspace{0.6em}
Practical note: the core results here were achieved with \textbf{minimal optimization}:
\begin{itemize}
  \item bin boundaries were generated once (not heavily tuned)
  \item the value comes from the architecture + training objective, not brittle heuristics
\end{itemize}
\end{frame}

\begin{frame}{Point estimate vs PDF: what changes in practice?}
\textbf{A point estimate answers:} ``what number should I predict?''
\vspace{0.3em}

\textbf{A PDF answers:} ``how much probability mass lies in each outcome region?''
\vspace{0.6em}

That enables:
\begin{itemize}
  \item tail-aware decisions (thresholds, risk budgets, intervention timing)
  \item calibrated confidence (wide vs narrow PDFs)
  \item multiple decision-relevant summaries (mean/median/mode/quantiles)
\end{itemize}
\end{frame}

\begin{frame}{Monte Carlo trajectory simulation (cone of uncertainty)}
Once the model outputs a PDF at each step, you can simulate plausible futures by \textbf{recursive sampling}.
\vspace{0.6em}

\textbf{At step \(t+1\)} sample:
\[
\tilde{y}_{t+1} \sim P(y \mid X_{\le t})
\]

\vspace{0.4em}
Repeat this rollout for \(M\) samples to get a \textbf{family of trajectories}
\(\{\mathcal{T}^{(m)}\}_{m=1}^{M}\), which forms a cone of uncertainty over long horizons.
\vspace{0.6em}

\textbf{What you can answer:}
\begin{itemize}
  \item 80\% interval (or any quantile band) for the state at a future horizon
  \item probability of crossing a threshold (tail mass) within the next \(k\) steps
\end{itemize}
\end{frame}

\begin{frame}{Token order is a hidden hyperparameter (generative rollouts)}
Auto-regressive generation defines a factorization:
\[
P(x_{1:n}) = \prod_{i=1}^{n} P(x_i \mid x_{<i})
\]
\vspace{0.4em}
\begin{itemize}
  \item If you generate a full future event block token-by-token, the \textbf{token ordering} implicitly chooses which conditionals you model (e.g., \(P(\text{temp}\mid\text{dist},\ldots)\) vs \(P(\text{dist}\mid\text{temp},\ldots)\)).
  \item To avoid incoherent ``stories'' in rollouts:
    \begin{itemize}
      \item \textbf{Conditional generation (recommended)}: treat future covariates (distance, weather) as given; only sample the target (pace).
      \item \textbf{Joint generation (if needed)}: decide what is exogenous vs endogenous and design/test token order as an explicit modeling choice.
    \end{itemize}
\end{itemize}
\end{frame}

\begin{frame}{Seconds-aware Gaussian smoothing (targets)}
Instead of ``hard'' one-hot targets, construct a soft target distribution \(T_i\) by integrating a Gaussian kernel over each bin interval:
\[
T_i = \int_{b_i^{start}}^{b_i^{end}} \frac{1}{\sigma\sqrt{2\pi}}
e^{-\frac{1}{2}\left(\frac{x-y_{true}}{\sigma}\right)^2}\, dx
\]
\vspace{0.6em}
This yields:
\begin{itemize}
  \item \textbf{ordinal awareness}: near misses are rewarded
  \item smoother training dynamics + better calibrated PDFs
\end{itemize}
\end{frame}

\begin{frame}{Benchmark setup (high level)}
\begin{itemize}
  \item evaluation split is \textbf{entity-disjoint} (unseen runners)
  \item report MAE in \textbf{seconds per mile}
  \item compare against simple baselines + XGBoost tabular regressor
\end{itemize}
\vspace{0.6em}
\textbf{XGBoost baseline (design choices)}: engineered pace stats (mean/EMA/min/max/volatility), cadence features, binned weather/context features, \textbf{continuous} distance\_miles from distance token, \textbf{one-hot} conditions, early stopping; artifacts written to an output directory.
\end{frame}

\begin{frame}{Continuous vs discretized: ``resolution disadvantage''}
\begin{itemize}
  \item \textbf{XGBoost} regresses directly to the \textbf{continuous} target (pace in seconds).
  \item \textbf{RunTime} predicts a \textbf{PDF over discretized pace bins} (\(\sim 270+\) classes); point estimates come from mean/median/mode over bin medians.
  \item \textbf{Chronos-2 comparison}: values remain \textbf{discretized tokens}, while irregular gaps are represented via \textbf{continuous time embeddings} (normalized time deltas).
  \item Discretization is both a \textbf{constraint} and a \textbf{regularizer}: it reduces sensitivity to micro-noise and helps avoid collapsing to a safe average.
  \item Despite this ``resolution disadvantage'', RunTime remains competitive with (or outperforms) the tuned tabular baseline.
\end{itemize}
\end{frame}

\begin{frame}{Baseline signal hierarchy (what matters most)}
\begin{itemize}
  \item In the XGBoost baseline, \textbf{historical performance} (EMA + distance-specific averages) dominates feature importance (\(>75\%\)).
  \item Environmental features (e.g., temperature/humidity) tend to contribute little (\(<2\%\)) relative to history.
  \item This motivates RunTime's core thesis: modeling \textbf{system rhythm} (history + cadence) is the main driver of accuracy.
\end{itemize}
\end{frame}

\begin{frame}{Results (runner-disjoint MAE)}
\small
\begin{table}
\centering
\begin{tabular}{l r}
\toprule
\textbf{Model} & \textbf{MAE (s)} \\
\midrule
Naive Mean & 54.19 \\
Last Race Pace & 61.31 \\
Riegel Formula & 50.76 \\
XGBoost & 40.94 \\
Transformer (Mean) & 37.67 \\
\textbf{Transformer (Median)} & \textbf{37.10} \\
Transformer (Mode) & 38.64 \\
\bottomrule
\end{tabular}
\end{table}
\normalsize
\vspace{0.3em}
\textbf{Improvement vs XGBoost (median)}: \(1 - 37.10/40.94 \approx 9.35\%\) MAE reduction.
\end{frame}

\begin{frame}{How performance scales with history}
\begin{center}
\includegraphics[width=0.92\linewidth]{../figures/mae_vs_experience.png}
\end{center}
\end{frame}

\begin{frame}{Training curves (W\&B exports)}
\begin{center}
\includegraphics[width=0.95\linewidth]{../figures/wandb-figures/val_train_plots.png}
\end{center}
\end{frame}

\begin{frame}{Validation curves (W\&B exports)}
\begin{center}
\includegraphics[width=0.95\linewidth]{../figures/wandb-figures/val_mae_loss.png}
\end{center}
\end{frame}

\begin{frame}{Percentile calibration histogram}
\begin{center}
\includegraphics[width=0.75\linewidth]{../figures/pace_percentiles.png}
\end{center}
Each bin spans 5 percentile points; n=10,000 predictions. Under perfect calibration every bin would hold ~500 entries, and the near-uniform counts (34–69) show the PDF percentiles are well balanced before deeper diagnostics.
\end{frame}

\begin{frame}{Quantile-quantile diagnostics}
\setlength{\columnsep}{0pt}
 \begin{columns}[T]
  \begin{column}{0.45\linewidth}
  {\small
Q-Q plot: KS \(D=0.025\), \(p<0.001\); max deviation approximately 2.5pp.
\vspace{0.3em}
The straight diagonal alignment against Uniform(0,1) proves the distributional predictions stay balanced across bins.
\vspace{0.3em}
Deviations remain small even at the upper tail, explaining the low KS statistic.
  }
  \end{column}
  \begin{column}{0.55\linewidth}
  \includegraphics[width=0.96\linewidth,height=0.79\textheight,keepaspectratio]{../figures/qq.png}
  \end{column}
\end{columns}
\end{frame}

\begin{frame}{Calibration curve}
 \begin{columns}[T]
  \begin{column}{0.45\linewidth}
  {\small
Calibration curve: 50,000 threshold predictions keep deviations under 3 percentage points while the mean predicted (60.1\%) matches the observed (59.7\%).
\vspace{0.3em}
Every bin hugs the diagonal, so probability mass at each confidence level is trustworthy.
\vspace{0.3em}
This stability justifies acting on any quantile without systematic bias.
  }
  \end{column}
  \begin{column}{0.55\linewidth}
  \includegraphics[width=\linewidth]{../figures/calibration.png}
  \end{column}
\end{columns}
\end{frame}

\begin{frame}{Interpretability: RY runner PDFs}
\begin{columns}[T]
\begin{column}{0.45\linewidth}
RY (n=9) shows moderate uncertainty. Shorter races yield tight PDFs (~45-58\%ile), while the marathon (\#9) becomes right-skewed with the actual outcome near 73.7\%ile, capturing distance-sensitive volatility without explicit distance modeling. The distributions therefore confirm that context-specific uncertainty pathways arise naturally from the tokenized input.
\end{column}
\begin{column}{0.55\linewidth}
\includegraphics[width=\linewidth]{../figures/RY.png}
\end{column}
\end{columns}
\end{frame}

\begin{frame}{Interpretability: RW runner PDFs}
\begin{columns}[T]
\begin{column}{0.45\linewidth}
RW (n=13) has the widest PDFs (0.015-0.025 peak), especially for marathons (16-100\%ile). Wide, asymmetric distributions and extreme outcomes such as race \#10 (100\%ile) and \#13 (15.6\%ile) still sit in the tails, illustrating that the model quantifies the athlete's inherent variability instead of collapsing to overconfident point estimates.
\end{column}
\begin{column}{0.55\linewidth}
\includegraphics[width=\linewidth]{../figures/RW.png}
\end{column}
\end{columns}
\end{frame}

\begin{frame}{Mechanistic inspection (example view)}
\begin{center}
\includegraphics[width=0.94\linewidth]{../figures/activation_plots/seed42_batch0_MEDIUM_middle_15__i1_Dan_h3_w80103_bylayer_20260121_070521.png}
\end{center}
\vspace{-0.4em}
\small
By-layer token-type attention helps validate \emph{what} the model uses (cadence vs context vs history) as a function of depth.
\normalsize
\end{frame}

\begin{frame}{Generalization beyond this dataset}
While motivated by irregular event sequences, the recipe applies broadly to regression and forecasting:
\begin{itemize}
  \item define a tokenization (or block) for context
  \item choose an output distribution (discretized bins)
  \item train for distributional prediction (PDF) instead of point prediction
\end{itemize}
\vspace{0.6em}
\textbf{Not just irregular events:}
\begin{itemize}
  \item \textbf{h=0 (no history)}: predict from static covariates only (standard regression)
  \item \textbf{regular intervals}: apply on fixed grids (standard time series) and still output PDFs
\end{itemize}
\vspace{0.4em}
This positions RunTime-style models as competitors to tabular baselines and classical forecasting (e.g., ARIMA), with the same distributional output interface.
\end{frame}

\begin{frame}{Immediate follow ups (from the paper)}
\small
Follow-ups to isolate which components drive the gains and validate generality beyond NYRR:
\begin{itemize}
  \item \textbf{Time-token ablation}: remove \texttt{d\_next} / \texttt{d\_fin} (or replace with a constant) to test performance without explicit gap knowledge
  \item \textbf{Swapped-token grammar}: pace already precedes cadence in the production stride to prevent leakage; we keep the original order as an ablation.
  \item \textbf{Zero-history baseline (h=0)}: static covariates only under the same entity-disjoint split
  \item \textbf{External validation}: apply to MIMIC-IV to predict a time-to-next-admission PDF (patient-disjoint split)
\end{itemize}
\normalsize
\end{frame}

\begin{frame}{Future application 1: system monitoring (health / maintenance)}
\textbf{Goal:} move from static thresholds to individualized trajectory-aware risk.
\vspace{0.5em}
\begin{itemize}
  \item tokens encode irregular measurements + context (state, environment, interventions)
  \item output is a \textbf{PDF over next measurement} (or time-to-event via target inversion)
  \item enables \textbf{early warnings} when observed values fall in low-probability regions
\end{itemize}
\vspace{0.4em}
\textbf{Why PDF matters:} decisions often depend on tail risk (rare but costly excursions).
\end{frame}

\begin{frame}{Future application 2: engagement / churn}
\textbf{Goal:} detect rhythm shifts before the terminal event (churn, dropout, relapse).
\vspace{0.5em}
\begin{itemize}
  \item events: sessions, purchases, support tickets, refills, check-ins
  \item cadence tokens capture \textbf{inter-arrival changes} and recency patterns
  \item output PDF can target outcome (e.g., next action) or time-to-next action
\end{itemize}
\vspace{0.4em}
\textbf{Why PDF matters:} interventions can be triggered when probability mass shifts into risky tails.
\end{frame}

\begin{frame}{Future application 3: hydrology \& energy (extremes)}
\textbf{Goal:} forecast distributions over extremes, not just averages.
\vspace{0.5em}
\begin{itemize}
  \item hydrology: PDFs over river height / levee-cross risk under irregular storm cadence
  \item energy: PDFs over localized peak demand / transformer stress during heat events
  \item evaluate by tail metrics (coverage at high quantiles), not only mean error
\end{itemize}
\vspace{0.4em}
\textbf{Why PDF matters:} infrastructure cost is dominated by rare peaks and failures.
\end{frame}

\begin{frame}{Future application 4: markets (tail-risk forecasting)}
\textbf{Goal:} predict distributions over drawdowns, volatility spikes, and jumps.
\vspace{0.5em}
\begin{itemize}
  \item events: trades, regime changes, macro shocks; cadence can be irregular
  \item output PDF can target \textbf{drawdown size} or \textbf{time to next shock}
  \item supports stress testing and risk budgeting via quantiles/ES-like summaries
\end{itemize}
\vspace{0.4em}
\textbf{Why PDF matters:} tail outcomes dominate hedging and capital requirements.
\end{frame}

\begin{frame}{Practical note on data availability}
To prevent abuse (e.g., automated scraping / bulk pulling of underlying raw results), \textbf{not all acquisition / raw-data retrieval steps are included} in the public repo.
\vspace{0.6em}
\begin{itemize}
  \item the repo remains runnable end-to-end on included sample shards
  \item interested researchers can reach out for additional details as appropriate
\end{itemize}
\end{frame}

\begin{frame}{Repro \& pointers}
\begin{itemize}
  \item Technical writeup: \texttt{Technical\_Details.md}
  \item Baselines: \texttt{train/Benchmark\_Baselines.py} (wrapper: \texttt{train/run\_xgboost\_tuning.sh})
  \item Evaluation notebook: \texttt{train/Inspect\_Model\_Outputs.ipynb}
  \item Activation inspection: \texttt{train/Inspect\_Model\_Activations.ipynb}
\end{itemize}
\vspace{0.8em}
\textbf{Repo}: \repo
\end{frame}

\begin{frame}
\centering
\Huge Questions?
\vspace{0.8em}
\par\large \repo
\end{frame}

\end{document}


